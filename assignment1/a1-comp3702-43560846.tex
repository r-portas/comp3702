\title{COMP3702 - Assignment 1}
\author{Roy Portas - 4356084}
\date{\today}

\documentclass[12pt]{article}
\usepackage{geometry}
\geometry{
    a4paper,
    margin=1in
}

\begin{document}
    \maketitle

    \section{Formal Definition of a Navigation Agent}
    The navigation agent in this assignment is designed to output the shorted route between two points.\\
    The navigation agent achieves this by gathering information about the environment, which in this case is the adjacency matrix))\\
    The navigation agents uses this data of the environment to find the shortest route\\
    \\
    The navigation agent can be divided into three sections:\\
    \begin{enumerate}
        \item Action space\\
            The set of all actions the agent can do, which in this cause is all possible paths between every point

        \item Percept space\\
            The set of all things the agent can perceive in the world, which in this case is all the points and the costs between every point

        \item State space\\
            The internal state of the world and the agent
    \end{enumerate}

    The navigation agent will use the above the select an action it believes will minimize the cost between two points.

    \section{Characteristics of Navigation Agent}

    The navigation agent will be discrete, because the points and paths between them are  finite.\\
    \\
    The agent is deterministic, because the agent knows exactly what state it will be in after moving along a path\\
    \\
    The agent is fully observable, because the agent knows the entire state of the world at any time. Additionally there are no external factors that could change the state\\
    \\
    The agent is static, because the world does not change while the agent is determining the  shortest path.

    \section{A* Search Heuristic}

    

    \section{Comparision of Uniform Cost and A*}
    \section{Algorithm scaling with input size}
\end{document}
