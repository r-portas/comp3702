\title{COMP3702 - Assignment 2}
\author{Roy Portas - 43560846}
\date{\today}

\documentclass[12pt]{article}
\usepackage{geometry}
\geometry{
    a4paper,
    margin=1in
}

\begin{document}
    \maketitle

    \section{Chosen Configuration Space}

    The maximum size of each chair movement is 0.001 units. The overall area is 1 unit squared. Thus if we choose the configuration space to be 0.001 increments, this will yield a 1000 by 1000 grid. 
    
    The area of this grid is 1000000, which should be viable for algorithm to use. Furthermore, choosing a smaller increment would not improve the accuracy, as the chair cannot move any finer than 0.001.
    
    However generating a grid of 1000000 items will likely take time and a lot of the workspace will be empty (not containing an obstacle). With that in mind, instead of generating a complete grid a sampling strategy should be used to focus on the points near obstacles.


    \section{Sampling Strategy}

    There are various sampling strategies that could work for the given problem, most notably:

    \begin{itemize}
        \item Sampling Near Obstacles
        
            This method is promising, since we are given all the obstacles at the start. Thus this sampling only needs to be done once. However since we are given the complete workspace, there are better choices for this problem.

        \item Sample Inside a Passage

            The obstacles for our problem can be sparse, with large distances between obstacles. Thus this strategy would not be an optimal choice.

        \item Using Workspace Information

            This strategy also looks promising, as we are given the complete workspace. Thus a possible solution is to sample the workspace randomly to create a discrete grid. The grid can then be converted into a search graph using rapidly exploring random trees.
    \end{itemize}

    \section{What will cause the program to succeed}

    \section{What will cause the program to fail}
\end{document}
