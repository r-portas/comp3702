\title{COMP3702 - Assignment 3}
\author{Roy Portas - 43560846}
\date{\today}

\documentclass[12pt]{article}
\usepackage{graphicx}
\usepackage{geometry}
\usepackage{listings}
\usepackage{verbatim}
\geometry{
    a4paper,
    margin=1in
}

\begin{document}
    \maketitle

    \section{Problem Definition}

    \subsection{State Space}
    The state space will be a set of tuples of integers from 0 to the maximum number of items the shop can stock. The size of the tuple is the number of items a shop can stock. 
    Thus for a tiny store it will be $\left\{(0, 0), (0, 1), (1, 0), \ldots, (3, 3)\right\}$.

    \subsection{Action Space}
    The actions space will be the set of all actions that can be performed by the customers. 
    Thus for a tiny store, it will look like the following:

    \begin{verbatim}
        {
            buy 0 of item 1,
            buy 1 of item 1,
            ...,
            buy 2 of item 2,
            buy 3 of item 2
        }
    \end{verbatim}

    \subsection{Transition Function}
    The transition function will be a matrix for each item type where the rows and columns are the number of items the store stocks.

    \begin{center}
        \begin{tabular} {|c|c|c|c|}
            \hline
            Item 1 & (0, 0) & (0, 1) & (1, 0) \\
            \hline
            (0, 0) & 0.2    & 0.3    & 0.1 \\
            (0, 1) & 0.2    & 0.3    & 0.1\\
            \hline
        \end{tabular}
    \end{center}

    \subsection{Reward Function}

\end{document}
